\documentclass[12pt]{article}
\usepackage[utf8]{inputenc}
\usepackage[spanish]{babel}
\usepackage{graphicx}
\usepackage{amsmath}
\usepackage{amssymb}
\usepackage{array}
\usepackage{xcolor}
\usepackage{geometry}
\usepackage{fancyhdr}
\usepackage{lastpage}
\usepackage{booktabs}
\usepackage{colortbl}
\usepackage{caption}
\usepackage{multirow}
\geometry{margin=1in}

\definecolor{lightblue}{RGB}{200,230,255}
\definecolor{lightgreen}{RGB}{220,255,220}
\definecolor{lightred}{RGB}{255,220,220}
\definecolor{lightyellow}{RGB}{255,255,200}
\pagestyle{fancy}
\fancyhf{}
\fancyhead[L]{Algoritmo de Floyd - Solución}
\fancyhead[R]{\thepage\ de \pageref{LastPage}}
\renewcommand{\headrulewidth}{0.4pt}
\renewcommand{\footrulewidth}{0.4pt}

\title{Algoritmo de Floyd - Solución}
\author{Emily Sanchez \\ Viviana Vargas}
\date{\today}

\begin{document}

\maketitle
\thispagestyle{empty}
\newpage
\setcounter{page}{1}

\section{Introducción}
El algoritmo de Floyd-Warshall es un algoritmo para encontrar los caminos más cortos en un grafo ponderado. Fue publicado por Robert Floyd en 1962.\\
\textbf{Complejidad temporal:} $O(n^3)$\\
\textbf{Complejidad espacial:} $O(n^2)$

\section{Descripción del Problema}
Grafo con 5 nodos:

\begin{itemize}
\item Nodo A: A
\item Nodo B: B
\item Nodo C: C
\item Nodo D: D
\item Nodo E: E
\end{itemize}

\section{Procedimiento del Algoritmo}
\subsection{Matriz de Distancias Inicial D(0)}
\begin{table}[h!]
\centering
\begin{tabular}{|c|c|c|c|c|c|}
\hline
 & A & B & C & D & E \\\hline
A & 0 & 6 & $\infty$ & 4 & 7 \\\hline
B & 9 & 0 & 7 & $\infty$ & $\infty$ \\\hline
C & $\infty$ & 5 & 0 & $\infty$ & 14 \\\hline
D & 8 & 1 & $\infty$ & 0 & 15 \\\hline
E & 2 & $\infty$ & 2 & 19 & 0 \\\hline
\end{tabular}
\caption{Matriz de distancias inicial D(0)}
\end{table}

\subsection{Matriz de Caminos Inicial P(0)}
\begin{table}[h!]
\centering
\begin{tabular}{|c|c|c|c|c|c|}
\hline
 & A & B & C & D & E \\\hline
A & - & A & - & A & A \\\hline
B & B & - & B & - & - \\\hline
C & - & C & - & - & C \\\hline
D & D & D & - & - & D \\\hline
E & E & - & E & E & - \\\hline
\end{tabular}
\caption{Matriz de caminos inicial P(0)}
\end{table}

\subsection{Iteraciones del Algoritmo}
\subsubsection{Iteración 1 (k = 1) - Nodo intermedio: A}
\begin{table}[h!]
\centering
\begin{tabular}{|c|c|c|c|c|c|}
\hline
 & A & B & C & D & E \\\hline
A & 0 & 6 & $\infty$ & 4 & 7 \\\hline
B & 9 & 0 & 7 & \cellcolor{lightgreen} 13 & \cellcolor{lightgreen} 16 \\\hline
C & $\infty$ & 5 & 0 & $\infty$ & 14 \\\hline
D & 8 & 1 & $\infty$ & 0 & 15 \\\hline
E & 2 & \cellcolor{lightgreen} 8 & 2 & \cellcolor{lightgreen} 6 & 0 \\\hline
\end{tabular}
\caption{Matriz de distancias D(1) - Cambios resaltados en verde}
\end{table}

\subsubsection{Iteración 2 (k = 2) - Nodo intermedio: B}
\begin{table}[h!]
\centering
\begin{tabular}{|c|c|c|c|c|c|}
\hline
 & A & B & C & D & E \\\hline
A & 0 & 6 & \cellcolor{lightgreen} 13 & 4 & 7 \\\hline
B & 9 & 0 & 7 & 13 & 16 \\\hline
C & \cellcolor{lightgreen} 14 & 5 & 0 & \cellcolor{lightgreen} 18 & 14 \\\hline
D & 8 & 1 & \cellcolor{lightgreen} 8 & 0 & 15 \\\hline
E & 2 & 8 & 2 & 6 & 0 \\\hline
\end{tabular}
\caption{Matriz de distancias D(2) - Cambios resaltados en verde}
\end{table}

\subsubsection{Iteración 3 (k = 3) - Nodo intermedio: C}
\begin{table}[h!]
\centering
\begin{tabular}{|c|c|c|c|c|c|}
\hline
 & A & B & C & D & E \\\hline
A & 0 & 6 & 13 & 4 & 7 \\\hline
B & 9 & 0 & 7 & 13 & 16 \\\hline
C & 14 & 5 & 0 & 18 & 14 \\\hline
D & 8 & 1 & 8 & 0 & 15 \\\hline
E & 2 & \cellcolor{lightgreen} 7 & 2 & 6 & 0 \\\hline
\end{tabular}
\caption{Matriz de distancias D(3) - Cambios resaltados en verde}
\end{table}

\subsubsection{Iteración 4 (k = 4) - Nodo intermedio: D}
\begin{table}[h!]
\centering
\begin{tabular}{|c|c|c|c|c|c|}
\hline
 & A & B & C & D & E \\\hline
A & 0 & \cellcolor{lightgreen} 5 & \cellcolor{lightgreen} 12 & 4 & 7 \\\hline
B & 9 & 0 & 7 & 13 & 16 \\\hline
C & 14 & 5 & 0 & 18 & 14 \\\hline
D & 8 & 1 & 8 & 0 & 15 \\\hline
E & 2 & 7 & 2 & 6 & 0 \\\hline
\end{tabular}
\caption{Matriz de distancias D(4) - Cambios resaltados en verde}
\end{table}

\subsubsection{Iteración 5 (k = 5) - Nodo intermedio: E}
\begin{table}[h!]
\centering
\begin{tabular}{|c|c|c|c|c|c|}
\hline
 & A & B & C & D & E \\\hline
A & 0 & 5 & \cellcolor{lightgreen} 9 & 4 & 7 \\\hline
B & 9 & 0 & 7 & 13 & 16 \\\hline
C & 14 & 5 & 0 & 18 & 14 \\\hline
D & 8 & 1 & 8 & 0 & 15 \\\hline
E & 2 & 7 & 2 & 6 & 0 \\\hline
\end{tabular}
\caption{Matriz de distancias D(5) - Cambios resaltados en verde}
\end{table}

\section{Resultados Finales}
\subsection{Matriz de Distancias Final D(5)}
\begin{table}[h!]
\centering
\begin{tabular}{|c|c|c|c|c|c|}
\hline
 & A & B & C & D & E \\\hline
A & 0 & 5 & 9 & 4 & 7 \\\hline
B & 9 & 0 & 7 & 13 & 16 \\\hline
C & 14 & 5 & 0 & 18 & 14 \\\hline
D & 8 & 1 & 8 & 0 & 15 \\\hline
E & 2 & 7 & 2 & 6 & 0 \\\hline
\end{tabular}
\caption{Matriz de distancias final D(5)}
\end{table}

\subsection{Rutas Óptimas}
\begin{itemize}
\item \textbf{A → B:} Distancia: 5, Ruta: A → D → B
\item \textbf{A → C:} Distancia: 9, Ruta: A → E → C
\item \textbf{A → D:} Distancia: 4, Ruta: A → D
\item \textbf{A → E:} Distancia: 7, Ruta: A → E
\item \textbf{B → A:} Distancia: 9, Ruta: B → A
\item \textbf{B → C:} Distancia: 7, Ruta: B → C
\item \textbf{B → D:} Distancia: 13, Ruta: B → A → D
\item \textbf{B → E:} Distancia: 16, Ruta: B → A → E
\item \textbf{C → A:} Distancia: 14, Ruta: C → B → A
\item \textbf{C → B:} Distancia: 5, Ruta: C → B
\item \textbf{C → D:} Distancia: 18, Ruta: C → B → A → D
\item \textbf{C → E:} Distancia: 14, Ruta: C → E
\item \textbf{D → A:} Distancia: 8, Ruta: D → A
\item \textbf{D → B:} Distancia: 1, Ruta: D → B
\item \textbf{D → C:} Distancia: 8, Ruta: D → B → C
\item \textbf{D → E:} Distancia: 15, Ruta: D → E
\item \textbf{E → A:} Distancia: 2, Ruta: E → A
\item \textbf{E → B:} Distancia: 7, Ruta: E → C → B
\item \textbf{E → C:} Distancia: 2, Ruta: E → C
\item \textbf{E → D:} Distancia: 6, Ruta: E → A → D
\end{itemize}
\end{document}
